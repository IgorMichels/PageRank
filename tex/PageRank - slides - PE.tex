\documentclass{beamer}
\usetheme{Ilmenau}
\usepackage[portuguese]{babel}

\usepackage{tikz}

\tikzstyle{node} = [circle, draw = blue!50, fill = blue!20, thick, inner sep = 0pt, minimum size = 6mm]

\title{Page Rank e Cadeias de Markov}
\author{
Cristhian Grundmann\\
Hanna Rodrigues Ferreira \\
Igor Cortes Junqueira \\
Igor Patrício Michels }
\date{\today}
\centering

\usepackage{graphicx}

\begin{document}
\maketitle

% ---------------------------------------------
\begin{frame}{Introdução}
\begin{itemize}

\item Em 1995, Larry Page e Sergey Brin se conheceram na universidade de Stanford

\vspace{2pt}

\item E objetivando criar um mecanismo de busca que pudesse

\vspace{3pt}

\begin{center}
    "organizar as informações do mundo e torná-las universalmente acessíveis e úteis"
\end{center}

\vspace{2pt}

\item Desenvolveram um algoritmo que usava os links para determinar a importância de cada página da internet \cite{google}.

\vspace{2pt}

\item Ele ganhou força, virando o principal mecanismo de buscas

\begin{itemize}
    \item até então, as estratégias de calcular a relevância eram calculadas usando apenas os dados da própria página.
\end{itemize}

% o que poderia ser facilmente burlado e deixaria resultados pouco relevantes nas primeiras posições

% principalmente pelo grande crescimento da internet na época.

\end{itemize}
\end{frame}
% ---------------------------------------------
\begin{frame}{Cadeias de Markov}
\begin{itemize}

\item Cadeias de Markov são um modelo estocástico que descreve uma sequência de eventos onde a probabilidade dos mesmos só depende do estado anterior.

% podem ser aplicados em tempo discreto ou continuo;
% são usados para diversos processos como filas ou dinâmica populacional animal;

\item Processos de Markov são a base para simulações do tipo 'Monte Carlo Markov Chain'.

% podem ser usados para amostrar de distribuições de probabilidade complexas, com aplicações em diversas áreas.

\item podem ser representadas por uma 'matriz de transição', que descreve as probabilidades de transição de cada estado em particular para os demais possíveis estados.

% torna-se fácil manipular computacionalmente o processo. 

\item Também é possível obter a 'distribuição estacionária', que representa uma especie de 'equilíbrio' no processo a longo prazo.

% onde é conhecida a representatividade de cada estado no todo. 

% \begin{itemize}
%     \item Essas distribuições estacionárias existem se, e somente se, a Cadeia de Markov em questão é 'não periódica'.
    
%  o que pode ser entendido como não haver ciclos de forma a limitar as transições a múltiplos fixos de passos para cada estado.
    
% \end{itemize}

\end{itemize}
\end{frame}
% ---------------------------------------------
\begin{frame}{Cadeias de Markov: O teorema de Perron-Frobenius}

% O Teorema de Perron-Frobenius afirma, resumidamente, que uma matriz $d \times d$ de elementos positivos possui um autovalor positivo com multiplicidade algébrica um e é o maior autovalor em módulo. 

% No nosso caso, isso é particularmente útil por garantir que uma matriz de transição de estados possua um autovetor correspondente ao autovalor $1$, sendo justamente o estado estacionário do processo modelado, além de garantir a unicidade do mesmo.

    Seja $A$ uma matriz $d \times d$ de entradas positivas, \\ isto é $A_{ij}>0$ $\forall i, j = 1, \dots, d$. Então:
    
    \bigskip
    
    \begin{enumerate}
        \item $A$ possui um único autovetor $x$ de norma $1$, cujas componentes são, todas elas, positivas;
        \item o autovalor $\lambda_+$ associado ao autovetor $x$ é positivo e, para qualquer outro autovalor $\lambda \in \mathbb{C}$, temos que $|\lambda| < \lambda_+$;
        \item o autovalor $\lambda_+$ é simples;
    \end{enumerate}
    
% A prova dos itens desse teorema se dão, dentre outras abordagens, por álgebra linear. 

\end{frame}
% ---------------------------------------------
\begin{frame}{Page Rank}
\begin{itemize}

% Indo um pouco na contramão de algumas outras estratégias

\item O PageRank visa calcular a relevância de uma página por meio de fatores externos.

\item podemos usar um grafo para ilustrar uma pequena rede, onde os nós são os sites e as arestas direcionadas representam o site de origem tem um link para o site de destino.

\item É esperado que o site que receba a maior quantidade de links, maior grau de entrada, deve ser o mais relevante, mas e em caso de empate?

\end{itemize}
\end{frame}
% ---------------------------------------------
\begin{frame}{Page Rank}
\begin{itemize}

\item Em caso de empate, a ideia de Page e Brin foi ponderar os votos de acordo com a relevância 

\item se um site tem uma alta relevância, seu voto deve ter um peso maior que o voto de um site que não recebe link algum.

\bigskip

\begin{center}
    Exemplo:
Um site que recebe apenas um link, mas do G1, deve ser mais relevante que um site que recebe apenas um link de um site pessoal.
\end{center}

\end{itemize}
\end{frame}
% ---------------------------------------------
\begin{frame}{Conexão com a Cadeia de Markov}
\begin{itemize}

\item Podemos pensar no processo de um internauta ficar navegando na internet e trocar de sites por meio de links presentes no próprio site, com mesma probabilidade para cada link.

\vspace{2pt}

\item Dessa forma, se um site $S$ tem link para $n$ diferentes sites ($T_1$, $T_2$, $\dots$ e $T_n$), a probabilidade do internauta sair do site $S$ para os sites $T_i, i \in \{1, 2, \dots, n\}$ é igual a $\frac{1}{n}$ e é nula para qualquer outro site. 

% Note que temos uma modelagem concisa que representa a rede de modo simples resta, então, definir uma métrica para representar a relevância de um site.

\vspace{2pt}

%Pensando em Cadeias de Markov, e em sua evolução temporal, 

\item Uma boa métrica para representar a relevância de um site seja dada pela proporção do tempo que o internauta passa em cada site ao realizar um passeio aleatório pelos mesmos.

\end{itemize}
\bigskip\end{frame}
% ---------------------------------------------
\begin{frame}{Conexão com a Cadeia de Markov}
\begin{itemize}

\item Ou seja, considerando que relevância é maior para sites mais recorrentes e menor para os menos visitados. conlcuímos que:

\vspace{5pt}

\begin{enumerate}
    \item sites que são muito linkados tendem a aparecer mais vezes no passeio.
    \item sites que são linkados pelos mais recorrentes também tendem a aparecer mais vezes.
    \item sites pouco linkados e com links de sites menos recorrentes tendem a aparecer por menos tempo no passeio.
\end{enumerate}

\vspace{5pt}

% Portanto, estamos com uma métrica em que os sites mais bem colocados tenham votos com peso maior que os piores colocados.

\vspace{2pt}

\item Logo, a relevância pode ser calculada por meio do vetor estacionário da Cadeia de Markov definida pela rede.

\end{itemize}
\end{frame}
% ---------------------------------------------
\begin{frame}{Alguns problemas na modelagem}
\begin{itemize}


\item Vamos supor que uma sub-rede da nossa rede seja cíclica, ao executarmos o passeio aleatório e um internauta cair dentro dele, ele ficará no ciclo infinitamente.

% em $A$, por exemplo, ele ficará seguindo o caminho $A\to B\to C\to A$ infinitamente.

% ao entrar no ciclo, o tempo que o internauta permanece em cada site desse ciclo é $\frac{1}{3}$, o que nos daria igual relevância a cada um dos três sites do ciclo e relevância zero para os demais sites da rede, o que seria problemático.

\begin{figure}[H]
    \centering
    \begin{minipage}{.5\textwidth}
        \centering
        \begin{tikzpicture}
            \node (A) at (0, 0.85) [node] {A};
            \node (B) at (-0.5, 0) [node] {B};
            \node (C) at (0.5, 0) [node] {C};
            \draw [->, line width = 0.25mm] (A) -- (B);
            \draw [->, line width = 0.25mm] (B) -- (C);
            \draw [->, line width = 0.25mm] (C) -- (A);
        \end{tikzpicture}
        \caption{Exemplo de rede cíclica.}
        \label{graph1}
    \end{minipage}%
    \begin{minipage}{.5\textwidth}
        \centering
        \begin{tikzpicture}
            \node (D) at (2.5, 1.275) [node] {D};
            \node (E) at (2.5, 0.425) [node] {E};
            \node (F) at (2.5, -0.425) [node] {F};
            \node (G) at (3.5, 0.425) [node] {G};
            \draw [->, line width = 0.25mm] (D) -- (G);
            \draw [->, line width = 0.25mm] (E) -- (G);
            \draw [->, line width = 0.25mm] (F) -- (G);
        \end{tikzpicture}
        \caption{O nó G é um nó terminal.}
        \label{graph2}
    \end{minipage}
\end{figure}

\vspace{1pt}

\item Quando temos a aparição de um nó terminal. No passeio aleatório, se o internauta acessar o site $G$, ele permanece nele infinitamente.

% ou seja, teremos relevância $1$ para o site $G$ e zero para os demais, o que também seria um problema.

\end{itemize}
\end{frame}
% ---------------------------------------------
\begin{frame}{Solucionando os problemas}
\begin{itemize}

% Para driblar esses problemas podemos fazer algumas adaptações nas redes, alterando um pouco a estrutura e as probabilidades, mas visando uma maior coerência e correção desses problemas.

\item podemos fazer algumas adaptações nas redes, alterando um pouco a estrutura e as probabilidades:

\vspace{8pt}

\begin{itemize}
    \item quanto a estrutura, faremos a inserção de arestas de modo que o grafo fique completo, com todos os nós apontando para todos os outros nós.
    
    \vspace{5pt}
    
    \item e quanto a probabilidades, fixaremos um valor $p\in (0, 1)$ e a matriz de transição dessa nova cadeia será dada por
\[\tilde{M} = (1 - p)\cdot M + p\cdot \dfrac{I}{n},\] 
onde temos

\begin{itemize}
    \item $\tilde{M}$: a matriz de transição da nova cadeia;
        
    \item $M$: a matriz de transição da cadeia original;
        
    \item $I$: a matriz de uns de tamanho $n\times n$ e;
        
    \item $n$: o número de nós da rede.
\end{itemize}

\end{itemize}


\end{itemize}
\end{frame}
% ---------------------------------------------
\begin{frame}{Solucionando os problemas}
\begin{itemize}
\item Visualmente, podemos ilustrar os grafos anteriores após as alterações como na Figura \ref{graph12}. Na rede da esquerda, as arestas azuis representam as próprias arestas, enquanto as vermelhas representam grupos de arestas que se conectam a cada uma das arestas da outra parte do grafo, representada no Grafo.

\begin{figure}[H]
    \centering
    \begin{tikzpicture}
        \node (A) at (0, 0.85) [node] {A};
        \node (B) at (-0.5, 0) [node] {B};
        \node (C) at (0.5, 0) [node] {C};
        \node (Graph) at (0, 2) [node] {Grafo};
        \draw [->, line width = 0.25mm] (A) -- (B);
        \draw [<-, dashed, blue] (A) to[out = 180, in = 180] (B);
        \draw [->, line width = 0.25mm] (B) -- (C);
        \draw [<-, dashed, blue] (B) to[out = -90, in = -90] (C);
        \draw [->, line width = 0.25mm] (C) -- (A);
        \draw [<-, dashed, blue] (C) to[out = 0, in = 0] (A);
        \draw [->, line width = 0.25mm] (Graph) -- (A);
        \draw [<->, dashed, red] (C) to[out = 45, in = 45] (Graph);
        \draw [<->, dashed, red] (B) to[out = 135, in = 180] (Graph);
        \draw [<->, dashed, red] (A) to[out = 45, in = 0] (Graph);
    
        \node (D) at (2.5, 1.275) [node] {D};
        \node (E) at (2.5, 0.425) [node] {E};
        \node (F) at (2.5, -0.425) [node] {F};
        \node (G) at (3.5, 0.425) [node] {G};
        \draw [->, line width = 0.25mm] (D) -- (G);
        \draw [->, line width = 0.25mm] (E) -- (G);
        \draw [->, line width = 0.25mm] (F) -- (G);
        
        \draw [<->, dashed, blue] (D) -- (E);
        \draw [<->, dashed, blue] (E) -- (F);
        \draw [<->, dashed, blue] (D) to[out = 180, in = 180] (F);
        \draw [->, dashed, blue] (G) to[out = 90, in = 0] (D);
        \draw [->, dashed, blue] (G) to[out = 157.5, in = 22.5] (E);
        \draw [->, dashed, blue] (G) to[out = -90, in = 00] (F);
    \end{tikzpicture}
    \caption{Grafos anteriores após alteração descrita.}
    \label{graph12}
\end{figure}

% Uma interpretação dessa alteração é a de que o processo ocorre com o internauta, antes de sair do site, jogando uma moeda de probabilidade $p$ para sair cara. Saindo cara, o internauta vai na barra de endereços e digita o endereço de qualquer site (podendo inclusive digitar o endereço do site atual) com mesma probabilidade, acessando o site que foi digitado na barra de endereços. Já se a moeda der coroa, então o internauta entra em um dos links da página, novamente com a mesma probabilidade entre os links existentes.

% Podemos perceber que, com essas alterações, os problemas de ficarmos presos num ciclo ou num nó terminal acabam, pois temos uma probabilidade positiva de ir a outro nó da rede. Por fim, note também que essa alteração fez a matriz $\tilde{M}$ ser positiva o que, pelo Teorema de Perron-Frobenius, diz que essa matriz possui um estado estacionário $\pi$.

\end{itemize}
\end{frame}
% ---------------------------------------------
\begin{frame}{Exemplo}
    \begin{figure}[H]
    \centering
    \begin{tikzpicture}
        \node (0) at (2, 0) [node] {0};
        \node (1) at (0.9271 * 2/3, 2.8532 * 2/3) [node] {1};
        \node (2) at (-2.4271 * 2/3, 1.7634 * 2/3) [node] {2};
        \node (3) at (-2.4271 * 2/3, -1.7634 * 2/3) [node] {3};
        \node (4) at (0.9271 * 2/3, -2.8532 * 2/3) [node] {4};
        \draw [<->] (0) -- (1);
        \draw [<->] (0) -- (2);
        \draw [<->] (0) -- (3);
        \draw [->] (0) -- (4);
        \draw [<->] (1) -- (2);
        \draw [<->] (1) -- (3);
        \draw [->] (1) -- (4);
        \draw [<->] (2) -- (3);
        \draw [->] (2) -- (4);
        \draw [->] (3) -- (4);
    \end{tikzpicture}
    \caption{Exemplo de rede.}
    \label{exemplo}
\end{figure}
\end{frame}
% ---------------------------------------------
\begin{frame}[allowframebreaks]
        \frametitle{Referências}
\begin{thebibliography}{9}

\bibitem{google} Como nós começamos e onde estamos hoje. \textit{Google}. \url{https://about.google/our-story/}.

\bibitem{usp} O algoritmo PageRank do Google. \textit{Miguel Frasson - ICMC/USP}. \url{https://edisciplinas.usp.br/pluginfile.php/5790758/mod_resource/content/1/pagerank-estat.pdf}.

\bibitem{wiki} ``PageRank''. \textit{Wikipedia}. \url{https://en.wikipedia.org/wiki/PageRank}.

\bibitem{perron-frobenius} O Teorema de Perron-Frobenius e a Ausência de Transição de Fase em Modelos Unidimensionais da Mecânica Estatíıstica. \textit{Marcelo Richard Hilário - UFMG. Gastão Braga - UFMG}. \url{https://www.ime.usp.br/~map2121/2014/map2121/programas/perron-frobenius.pdf}.

\end{thebibliography}

\end{frame}
%------------------------------------------------
\end{document}

